\documentclass[12pt]{article}
\usepackage{graphicx}
\usepackage{enumitem}
\usepackage{amsthm}
\usepackage{amsmath}
\usepackage{amssymb}
\usepackage[utf8]{inputenc}
\usepackage{lmodern}
\usepackage[T1]{fontenc}
\usepackage{arcs}
\usepackage{etoolbox}
\usepackage{fancyhdr}
\usepackage{hyperref}
\usepackage{cleveref}
\renewcommand{\qed}{\hfill\square}
\numberwithin{equation}{subsection}
\newtheorem{axiom}{Axioma}
\numberwithin{axiom}{subsection}
\newtheorem{lemma}{Lema}
\numberwithin{lemma}{subsection}
\newtheorem*{claim}{Claim}
\newtheorem{theorem}{Teorema}
\numberwithin{theorem}{subsection}
\newtheorem{ruleinf}{Regla}
\numberwithin{ruleinf}{subsection}
%\usepackage[utf8]{inputenc}
%\usepackage[english]{babel}
\usepackage{tikz}
\usetikzlibrary{matrix}
\pagestyle{fancy}

\lhead{Compendio matemático}
%\lfoot{X}
%\lfoot{Vol I – Fundamentos y Axiomas}
%\rfoot{por Olemis Lang}

\begin{document}

\section{Metamatemáticas}

La metamatemática es una rama de la matemática dedicada a formalizar en términos matemáticos los elementos básicos de esta ciencia y sus relaciones. Esta sección no tiene como propósito una recopilación exhaustiva, profunda y sistemática  de hechos y teoremas en esta rama, Esto se debe a que el discurso vaa dirigido en buena medida a estudiantes de nivel pre-universitario, quizás secundaria. Se incluye estaa sección en función de entender los temas tratados posteriormente, algunos aspectos dde las demostraciones y brindar gerramientas de razonamiento al abordar la resolución de problemas, así como referencias para comprobar que son correctos.

\subsection{Fórmulas}

Las demostraciones se realizan de manera progresiva. Para decribir lo que sucede en cada paso se utiliza un lenguaje matemático universal (o sea, comprensible por cualquier persona más allá del lenguaje que se hable en su país de orígen o residencia).. El elemento sintáctico fundamental de este lenguaje son las fórmulas.

Lo primero que hay que coocer para comprender las fórmulas es el alfabeto (los símbolos) que se utilizan en su escritura. Como el tema es demasiado abarcador para ajustarnos al tema lo más posible y quue se comprenda fácilmente, es preciso centrarse en los símbolos solamente  Posteriormente se profundizará en los respectivos significados y las reglas para ubicar los símbolos con el fin de formar fórmulas correctas. En las fórmulas se pueden usar los siguientes símbolos:

\begin{itemize}
\item{\textbf{Variables:} que se representan con letras minúsculas tales como $a$, $b$, $c$, ... $x$, $y$, $z$, ... Es posible  utilizar también las letras griagas $\alpha$, $\beta$ , ... y demás.}
\item{\textbf{Mapeo o función} símbolo $\mapsto$ que se usa en una manera de definir funciones.}
\item{\textbf{Constantes lógicas} $\top$ (verdadero) y $\bot$ (falso).}
\item{\textbf{Conectores lógicos} que representan las siguientes operaciones lógicas:}
	\begin{itemize}
		\item{\textit{Negación} $\neg$ para expresar la no validez de una proposición lógica}
		\item{\textit{Conjunción} ($\land$) para expresar la validez simulánea de varias proposiciones lógicas}
		\item{\textit{Disjunción} ($\lor$) para expresar la validez de al menos unna de varias proposiciones lógicas}
		\item{\textit{Implicación} ($\implies$ o $\to$) para relaciones de causa y efecto}
		\item{\textit{Bicondicionalidad o equivalencia} ($\leftrightarrow$ o $\equiv$) para la implicación en ambos sentidos}
	\end{itemize}
\item{\textbf{Símbolo de igualdad} $=$.}
\item{\textbf{Símbolos matemáticos específicos:} También en las fórmulas pueden aperecer todos los símbolos específicos de las diferentes ramas de la matemática y se utilizan con el fin de expresar las particularidades de cada materia. En especial se pueden utilizar valores constantes como múmeros enteros($1$, $2$, $4$, ...), reales ($1.25$, ...) y otros según se definan para cada temática. Por ejemplo , en el unniverso del discurso de la teoría de conjuntos , los símbolos $\emptyset$ o $\varnothing$ representan indistintamente la noción del conjunto vacío.}
\item{\textbf{Símbolos de puntuación} que permiten agrupar, separar o dar legibilidad a la representación escrita de una fórmula. Por ejemplo, paréntesis, corchetes, llaves, comas, los dos puntos, y otros.}
\end{itemize}

Las fórmulas correctas tienen reglas sintácticas, o sea maneras de organizar los símbolos para que tengan sentido.y transmitan un significado. Hay dos grandes grupos de reglas como veremos a continuación. 

Los términos son las fórmulas más básicas y expresan hechos u objetos específicos de cada rama de la matemática.

\begin{itemize}
	\item{Las constantes (lógica o no).son términos. Estos se definen con el símbolo correspondiente.  Por ejemplo, $\emptyset$, $1$, $\top$ .}
	\item{Las variables tales como $x$, $u$, ... también son términos.}
	\item{Las expresiones matemáticas de los dominnios específicos también se consideran términos, e.g. $x^2 + 1$, $\lim_{t \to \infty}{t+2}$}
	\item{Las funciones son términos. Es posible escribirlas de la forma $f(t\textsubscript{1}, t\textsubscript{2}, ..., x\textsubscript{n})$ , donde cada variable $t\textsubscript{n}$ entre paréntesis representa otro término . Por ejemplo , $f(u,w^2+1)$, $\sin(x)$ son términos que representan funciones. Otra manera de escribir funciones es ubicando las variebles que representan parámetros a la izquierda de $\mapsto$ seguido de una expresión que define la función. Es decir , son términos válidos tanto $x \mapsto x^2+1$ como $x, y \mapsto x+y^4-5$.}
	\item{Todas las fórmulas  obtenidas por un número finito de aplicaciones sucesivas de las reglas anteriores, y solamente esas, son términos}
\end{itemize}

Las fórmulas complejas se definen inductivamente de la manera siguiente:

\begin{itemize}
	\item{Si $t\textsubscript{1}$ y $t\textsubscript{2}$ son términos , entonces $t\textsubscript{1}=t\textsubscript{2}$ es una fórmula que representa la igualdad.}
	\item{Si $\varphi$ es una fórmula entonces $\neg\varphi$ es una fórmula que representa la negación.}
	\item{Las expresiones en entrefijo formadas al colocar un conector lógico  entre otras dos fórmulas , también es una fórmula. Por ejemplo , si $\varphi$ y $\phi$ son fórmulas entonces $\varphi \land \phi$, $\varphi \lor \phi$, $\varphi \to \phi$, $\varphi \equiv \phi$, $\varphi \implies \phi$ también son fórmulas.}
	\item{Si $\varphi$ es una fórmula y $x$ una variable entonces las expresiones cuantificadas $\forall x\ \varphi$ (para todo $x$ se cumple $\varphi$) y $\exists x\ \varphi$ (existe $x$ tal que se cumple $\varphi$) también son fórmulas.}
	\item{Si $\varphi$ y $\phi$ son fórmulas y $x$ una variable , entonces $\varphi\left[x \backslash \phi\right]$ es una fórmula que se refiere a la sustitución de la variable $x$ por la fórmula $\phi$ en todas las posiciones dentro de la fórmula $\varphi$ donde aparezca $x$ como variable libre. Por ejemplo $(x^2+1)[x \backslash \sin(u)]$ es una una fórmula que hace referenci a $(\sin(u)^2+1)$}
\end{itemize}

Por convención se establecen las siguientes reglas de precedencia

\begin{itemize}
	\item{$\neg$ se evalúa primero en el orden en que aparezcan de derecha a izquierda}
	\item{$\land$ y $\lor$ se evalúan posteriormente en el orden en que aparezcan de izquierda a derecha}
	\item{$\forall$ y $\exists$ se evalúan posteriormente}
	\item{$\to$ se evalúa a continuación}
	\item{$\implies$ se evalúa finalmente}
\end{itemize}

Usando esas reglas se puede expresar cualquier proposición matemática. A continuación se muestran varias dórmulas a modo de ejemplificar el uso de las reglas y su relación con frases e el lenguaje natural.

\begin{itemize}
	\item{$\forall x,y \in \mathbb{R}\ \exists q \in \mathbb{Q}\ (x \leq q) \land (q \leq y)$ }
		\begin{itemize}
			\item{Entre todo par de números reales existe un número racional}
		\end{itemize}
	\item{$\forall A,B,C \in \mathbb{R}^2\ triangle(A, B, C) \to \exists c\ circle(c) \land (A \in c) \land (B \in c) \land (C \in c)$ }
		\begin{itemize}
			\item{Para todo trío de puntos en el plano no alinedos existe una circunferencia que pasa por los tres.}
		\end{itemize}
\end{itemize}

\subsection{Varibles libres y vinculadas}

En el epígrafe anterior se hizo referencia a las vvariables libres . Las mismas son las variables dentro de una fórmula que admiten algún tipo de sustitución. Brevemente explicaremos a continuación cómo se define este concepto.



\subsection{Reglas de inferencia}

Las demostraaciones matemáticas se basan en una sucesión de resultados intermedios que permiten llegar a una conclusión. Las fórmulas son útiles para describir los resultados obenidos en cada paso. Lasreglas de inferencia reflejan las maneras correctas de transformar la estructura de fórmulas que expresan hechos conocidos o demostrdos como válidos (antecedente) para arrivar a conclusiones que sean igualmente válidas (consecuente). Es por esto que pueden ser interpretadas como estrategias para demostrar o resolver un problema matemático.

Al abordar las transformaciones que se suceden durante una prueba formal resulta útil utilizar la notación secuente que se escribe de la siguiente maanera.

\begin{center}
$A\textsubscript{1}, A\textsubscript{2}, ... A\textsubscript{n} \vdash C$
\end{center}

Esta notación expresa que al asumir la veracidad de los antecedentes (fórmulas) $A\textsubscript{1}, A\textsubscript{2}, ... A\textsubscript{n}$ existe una sucesió de transformaciones de las mismas acordes a las reglas de inferencia, que permiten aseguraar la validez de la conclusión expresada en la fórmula $C$ .

Veamos cada regla a continuación para tener una idea más precisa. Comencemos por las más afines a la propia notación secuente para, quizás, entender mejor las implicaciones de su significado.

\begin{ruleinf} \label{relinf_impadd} Teorema de la deducción 
	\begin{equation}
	\begin{gathered}
		\underline {\varphi \vdash \psi }\\
		\varphi \to \psi 
	\end{gathered}
	\end{equation}
\end{ruleinf}


También conocida como \textit{introducción condicional}, es la regla que permite introducir el signo de implicación en la fórmula que queda como consecuente después de la transformación. En términos prácticos esta regla nos da una herramienta para la resolución de problemas, ya que de su planteamiento se constata que si se desea demostrar que \textit{"algo"} implica \textit{"otra cosa"} entonces se puede adoptar como estrategia asumir que ese \textit{"algo"} es verdadero y a partir de eso ir encadenando razonamientos intermedios (basados siempre en las reglas de inferencia) hasta llegar al punto en que la \textit{"otra cosa"} quede demostrada como verdadera. De lograr dicha secuencia de demostración la relación de implicación que buscamos se infiere respaldada por la aplicación directa de esta regla.


\subsection{Nociones de lógica}


\section{Las relaciones básicas}
\subsection{Igualdad}


\begin{axiom} \label{eq_reflexive} La relación de igualdad es reflexiva
	\begin{equation}
		\forall x\ x=x
	\end{equation}
\end{axiom}

\begin{axiom} \label{eq_subs} Propiedad de sustitución de la relación de igualdad
\begin{center}
	Sea $\phi$ un símbolo que denota una fórmula
\end{center}
	\begin{equation}
		\forall x,y,\phi\ (x=y) \implies (\phi\ = \phi[x \backslash y])
	\end{equation}
\end{axiom}




\begin{theorem} \label{eq_symm} La relación de igualdad es simétrica
	\begin{equation}
		\forall x,y\ (x=y) \implies (y=x)
	\end{equation}
\end{theorem}


\textit{Proof}. 

Es posible constatar que $y = x$ es una  posible sustitución parcial en \cref{eq_reflexive}., por lo tanto , según \cref{eq_subs} se obtiene que

\begin{equation} \label{eq_sym_proof_1}
\begin{gathered}
	\labelcref{eq_subs} , \quad \labelcref{eq_reflexive}\ x=x \quad \vdash \quad (x = y) \implies ((x = x) \implies (y=x))
\end{gathered}
\end{equation}

Basados en \cref{relinf_modustollens}  modus tollens

\begin{equation} \label{eq_sym_proof_2}
\begin{gathered}
	(x = x) \implies (y=x) \\
	\underline{\labelcref{eq_reflexive} \quad x=x \quad \quad \quad} \\
	y = x
\end{gathered}
\end{equation}

\begin{equation}
	\therefore \labelcref{eq_sym_proof_1} , \quad \labelcref{eq_sym_proof_2} \quad \vdash \quad (x = y) \implies (y = x)
\end{equation}

\hfill $\square$

\begin{theorem} \label{eq_trans} La relación de igualdad es transitiva
	\begin{equation}
		\forall x,y,z\ (x=y) \land (y=z) \implies (x=z)
	\end{equation}
\end{theorem}




\section{Los números}
\subsection{Números naturales}


Los elementos del conjunto 

\begin{axiom} \label{nat_zero}
	Elemento nulo
	\begin{equation}
		0 \in \mathbb{N}
	\end{equation}
\end{axiom}
\begin{axiom} \label{nat_closure}
	El conjunto de los números naturales es cerrado con respecto a la relación de sucesor.
	\begin{equation}
		x \in \mathbb{N}\ \implies S(x) \in \mathbb{N}
	\end{equation}
\end{axiom}


Para los números enteros se define una relación de igualdad que, por lo tanto, cumple con \cref{eq_reflexive} , \cref{eq_subs}. En el sistema de Peano \cref{eq_symm} y \cref{eq_trans} son considerados como axiomas. Sin embargo los mismos son demostrales a partir del principio de sustitución \cref{eq_subs}.



\begin{axiom} \label{nat_nosucc} No existe número  natural cuyo sucesor sea el elemento nulo 
	\begin{equation}
		\forall x \in \mathbb{N}\ (0\neq S(x))
	\end{equation}
\end{axiom}
\begin{axiom} \label{nat_inj} La relación de sucesor es inyectiva.
\begin{equation}
	\forall x,y  \in \mathbb{N}\ (S(x)=S(y)\implies x=y)
\end{equation}
\end{axiom}
\begin{axiom} \label{nat_eqclosurej} El conjunto de los números naturales es cerrado con respecto a la relación de igualdad
	\begin{equation}
		\forall x,y \ ((x \in \mathbb{N} \land x=y) \implies y )
	\end{equation}
\end{axiom}


\begin{axiom} \label{nat_sumz} El elemento nulo es la identidad de la operación de suma
	\begin{equation}
		\forall x  \in \mathbb{N}\ (x+0=x)
	\end{equation}
\end{axiom}
\begin{axiom} \label{nat_sumnz} Definición del operador de suma
	\begin{equation}
		\forall x,y  \in \mathbb{N}\ (x+S(y)=S(x+y))
	\end{equation}
\end{axiom}
\begin{axiom} \label{nat_mulz} Muliplicación por elementto nulo
	\begin{equation}
		\forall x  \in \mathbb{N}\ (x\cdot 0=0)
	\end{equation}
\end{axiom}
\begin{axiom} \label{nat_mulnz} Definición del operador de multiplicación
	\begin{equation}
		\forall x,y  \in \mathbb{N}\ (x\cdot S(y)=x\cdot y+x)
	\end{equation}
\end{axiom}


\begin{axiom} \label{nat_ind}
	Principio de inducción matemática
	\begin{center}
		Sea $\phi$ una fórmula con una variable libre $y$ ...
	\end{center}
	\begin{equation}
		{\bigg (}\phi \left[y\backslash0\right] \land \forall n \in \mathbb{N}{\Big (}\phi\left[y\backslash n\right] \implies \phi\left[y \backslash S(n)\right]{\Big )}{\bigg )} \implies \forall x \in \mathbf{N}\ \phi \left[y \backslash x\right]
	\end{equation}
\end{axiom}



\end{document}