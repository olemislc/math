\documentclass[12pt]{article}
\usepackage{graphicx}
\usepackage{enumitem}
\usepackage{amsthm}
\usepackage{amsmath}
\usepackage{amssymb}
\usepackage[utf8]{inputenc}
\usepackage{lmodern}
\usepackage[T1]{fontenc}
\usepackage{arcs}
\usepackage{etoolbox}
\usepackage{fancyhdr}
\usepackage{hyperref}
\usepackage{cleveref}
\renewcommand{\qed}{\hfill\square}
\numberwithin{equation}{subsection}
\newtheorem{axiom}{Axioma}
\numberwithin{axiom}{subsection}
\newtheorem{lemma}{Lema}
\numberwithin{lemma}{subsection}
\newtheorem*{claim}{Claim}
\newtheorem{theorem}{Teorema}
\numberwithin{theorem}{subsection}
\newtheorem{ruleinf}{Regla}
\numberwithin{ruleinf}{subsection}
%\usepackage[utf8]{inputenc}
%\usepackage[english]{babel}
\usepackage{tikz}
\usetikzlibrary{matrix}
\pagestyle{fancy}

\lhead{Compendio matemático}
%\lfoot{X}
%\lfoot{Vol I – Fundamentos y Axiomas}
%\rfoot{por Olemis Lang}

\begin{document}

\section{Metamatemáticas}

La metamatemática es una rama de la matemática dedicada a formalizar en términos matemáticos los elementos básicos de esta ciencia y sus relaciones. Esta sección no tiene como propósito una recopilación exhaustiva, profunda y sistemática  de hechos y teoremas en esta rama. Esto se debe a que el discurso va dirigido en buena medida a estudiantes de nivel pre-universitario, quizás secundaria. Se incluye este contenido en función de entender los temas tratados posteriormente, explicar el estilo en que se expresan algunos aspectos de las demostraciones y brindar herramientas de razonamiento al abordar la resolución de problemas, así como referencias para comprobar que son correctos.

\subsection{Fórmulas} \label{chp_formulae}

Las demostraciones se realizan de manera progresiva. Para decribir lo que sucede en cada paso se utiliza un lenguaje matemático universal (o sea, comprensible por cualquier persona más allá del lenguaje que se hable en su país de orígen o residencia).. El elemento sintáctico fundamental de este lenguaje son las fórmulas.

Lo primero que hay que coocer para comprender las fórmulas es el alfabeto (los símbolos) que se utilizan en su escritura. Como el ámbito de esta ciencia es demasiado abarcador, para ajustarnos al tema lo más posible y que se comprenda fácilmente, es preciso centrarse en los símbolos solamente  Posteriormente se profundizará en los respectivos significados y las reglas para ubicar los símbolos con el fin de formar fórmulas correctas. En las fórmulas se pueden usar los siguientes símbolos:

\begin{itemize}
\item{\textbf{Variables:} que se representan con letras minúsculas tales como $a$, $b$, $c$, ... $x$, $y$, $z$, ... Es posible  utilizar también las letras griegas $\alpha$, $\beta$ , ... y demás.}
\item{\textbf{Mapeo o función} símbolo $\mapsto$ que se usa en una manera de definir funciones.}
\item{\textbf{Constantes lógicas} $\top$ (verdadero) y $\bot$ (falso).}
\item{\textbf{Conectores lógicos} que representan las siguientes operaciones lógicas:}
	\begin{itemize}
		\item{\textit{Negación} $\neg$ para expresar la no validez de una proposición lógica}
		\item{\textit{Conjunción} ($\land$) para expresar la validez simulánea de varias proposiciones lógicas}
		\item{\textit{Disjunción} ($\lor$) para expresar la validez de al menos una de varias proposiciones lógicas}
		\item{\textit{Implicación} ($\implies$ o $\to$) para relaciones de causa y efecto}
		\item{\textit{Bicondicionalidad o equivalencia} ($\leftrightarrow$ o $\equiv$) para la implicación en ambos sentidos}
	\end{itemize}
\item{\textbf{Símbolo de igualdad} $=$.}
\item{\textbf{Símbolos matemáticos específicos:} También en las fórmulas pueden aparecer todos los símbolos específicos de las diferentes ramas de la matemática y se utilizan con el fin de expresar las particularidades de cada materia. En especial se pueden utilizar valores constantes como múmeros enteros($1$, $2$, $4$, ...), reales ($1.25$, ...) y otros según se definan para cada temática. Por ejemplo , en el universo del discurso de la teoría de conjuntos , los símbolos $\emptyset$ o $\varnothing$ representan indistintamente la noción del conjunto vacío.}
\item{\textbf{Símbolos de puntuación} que permiten agrupar, separar o dar legibilidad a la representación escrita de una fórmula. Por ejemplo, paréntesis, corchetes, llaves, comas, los dos puntos, y otros.}
\end{itemize}

Las fórmulas correctas tienen reglas sintácticas, o sea maneras de organizar los símbolos para que tengan sentido y transmitan un significado. Hay dos grandes grupos de reglas como veremos a continuación. 

Los términos son las fórmulas más básicas y expresan hechos u objetos específicos de cada rama de la matemática.

\begin{itemize}
	\item{Las constantes (lógica o no) son términos. Estos se definen con el símbolo correspondiente.  Por ejemplo, $\emptyset$, $1$, $\top$ .}
	\item{Las variables tales como $x$, $u$, ... también son términos.}
	\item{Las expresiones matemáticas de los dominios específicos también se consideran términos, e.g. $x^2 + 1$, $\lim_{t \to \infty}{t+2}$}
	\item{Las funciones son términos. Es posible escribirlas de la forma $f(t\textsubscript{1}, t\textsubscript{2}, ..., t\textsubscript{n})$ , donde cada variable $t\textsubscript{n}$ entre paréntesis representa otro término . Por ejemplo , $f(u,w^2+1)$, $\sin(x)$ son términos que representan funciones. Otra manera de escribir funciones es ubicando las variebles que representan parámetros a la izquierda de $\mapsto$ seguido de una expresión que define la función. Es decir , son términos válidos tanto $x \mapsto x^2+1$ como $x, y \mapsto x+y^4-5$.}
	\item{Todas las fórmulas  obtenidas por un número finito de aplicaciones sucesivas de las reglas anteriores, y solamente esas, son términos}
\end{itemize}

Las fórmulas complejas se definen inductivamente de la manera siguiente:

\begin{itemize}
	\item{Si $t\textsubscript{1}$ y $t\textsubscript{2}$ son términos , entonces $t\textsubscript{1}=t\textsubscript{2}$ es una fórmula que representa la igualdad.}
	\item{Si $\varphi$ es una fórmula entonces $\neg\varphi$ es una fórmula que representa la negación.}
	\item{Las expresiones en entrefijo formadas al colocar un conector lógico  entre otras dos fórmulas , también es una fórmula. Por ejemplo , si $\varphi$ y $\phi$ son fórmulas entonces $\varphi \land \phi$, $\varphi \lor \phi$, $\varphi \to \phi$, $\varphi \equiv \phi$, $\varphi \implies \phi$ también son fórmulas.}
	\item{Si $\varphi$ es una fórmula y $x$ una variable entonces las expresiones cuantificadas $\forall x\ \varphi$ (para todo $x$ se cumple $\varphi$) y $\exists x\ \varphi$ (existe $x$ tal que se cumple $\varphi$) también son fórmulas.}
	\item{Si $\varphi$ y $\phi$ son fórmulas y $x$ una variable , entonces $\varphi\left[x \backslash \phi\right]$ es una fórmula que se refiere a la sustitución de la variable $x$ para en su lugar insertar a $\phi$ en todas las posiciones dentro de la fórmula $\varphi$ donde aparezca $x$ como variable libre. Por ejemplo $(x^2+1)[x \backslash \sin(u)]$ es una una fórmula que hace referencia a $(\sin(u)^2+1)$}
\end{itemize}

Por convención se establecen las siguientes reglas de precedencia

\begin{itemize}
	\item{$\neg$ se evalúa primero en el orden en que aparezcan de derecha a izquierda}
	\item{$\land$ y $\lor$ se evalúan posteriormente en el orden en que aparezcan de izquierda a derecha}
	\item{$\forall$ y $\exists$ se evalúan posteriormente}
	\item{$\to$ y $\leftrightarrow$ se evalúan a continuación}
	\item{$\implies$ $\iff$ se evalúan finalmente}
\end{itemize}

Usando esas reglas se puede expresar cualquier proposición matemática. A continuación se muestran varias fórmulas a modo de ejemplificar el uso de las reglas y su relación con frases en el lenguaje natural.

\begin{itemize}
	\item{$\forall x,y \in \mathbb{R}\ \exists q \in \mathbb{Q}\ (x \leq q) \land (q \leq y)$ }
		\begin{itemize}
			\item{Entre todo par de números reales existe un número racional}
		\end{itemize}
	\item{$\forall A,B,C \in \mathbb{R}^2\ triangle(A, B, C) \to \exists c\ circle(c) \land (A \in c) \land (B \in c) \land (C \in c)$ }
		\begin{itemize}
			\item{Para todo trío de puntos en el plano no alineados existe una circunferencia que pasa por los tres.}
		\end{itemize}
\end{itemize}

\subsection{Variables libres y vinculadas}

En el epígrafe anterior se hizo referencia a las vvariables libres . Las mismas son las variables dentro de una fórmula que admiten algún tipo de sustitución. A continuación se ofrece una definición partiendo del concepto más abarcador hasta llegar al más específico, segú la definición de formula en \cref{chp_formulae}. Comencemos entonces por las reglas de definición de las fórmulas, desde las complejas hasta las más simples.

\begin{itemize}
	\item{En la fórmula de sustititución $\varphi\left[x \backslash \phi\right]$ , } ...
		\begin{itemize}
			\item{... una variable aparece libre si y solo si aparece libre en $\phi$ o aparece libre en $\varphi$ y es un símbolo diferente de la variable sustituida, en este caso $x$}
			\item{... una variable aparece vinculada si y solo si aparece vinculada en $\phi$ o aparece vinculada en $\varphi$ }
		\end{itemize}
	\item{En las expresiones cuantificadas $\forall x\ \varphi$ y $\exists x\ \varphi$ ...}
		\begin{itemize}
			\item{... una variable aparece libre si y solo si ocurre libre en $\varphi$ y es diferente al símbolo cuantificado, en el caso de referencia sería $x$}
			\item{... una variable aparece vinculada si  y solo si es el símbolo cuantificado, en el caso de referencia sería $x$, o por lo contrario ocurre vinculada en $\varphi$}
		\end{itemize}
	\item{En las expresiones lógicas formadas al colocar un conector lógico  entre otras dos fórmulas, tales como $\varphi \land \phi$, $\varphi \lor \phi$, $\varphi \to \phi$, $\varphi \equiv \phi$, $\varphi \implies \phi$ t...}
		\begin{itemize}
			\item{... una variable aparece libre si y solo si aparece libre en alguna de las fórmulas $\varphi$ o $\phi$}
			\item{... una variable aparece vinculada si  y solo si ocurre vinculada en alguna de las fórmulas $\varphi$ o $\phi$}
		\end{itemize}
	\item{En una negación $\neg\varphi$ ...}
		\begin{itemize}
			\item{... una variable aparece libre si y solo si ocurre libre en $\varphi$}
			\item{... una variable aparece vinculada si  y solo si ocurre vinculada en $\varphi$}
		\end{itemize}
	\item{En una igualdad $t\textsubscript{1}=t\textsubscript{2}$ ...}
		\begin{itemize}
			\item{... una variable aparece libre si y solo si ocurre libre en $t\textsubscript{1}$ o $t\textsubscript{2}$ }
			\item{... una variable aparece vinculada si  y solo si ocurre libre en $t\textsubscript{1}$ o $t\textsubscript{2}$ }
		\end{itemize}
	
	\item{Las constantes (lógica o no) no contienen, como expresión, variables ni libres ni vinculadas.}
	\item{Las fórmulas constituídas por variables tales como $x$, $u$, en todos los casos son ocurrencias libres y nunca vinculadas.}
	\item{En las expresiones matemáticas de los dominios específicos ...}
		\begin{itemize}
			\item{... una variable aparece vinculada si  y solo si es un símbolo de control de un notación matemática parametrizada.}
			\item{... una variable aparece libre en caso contrario}
			\item{Por ejemplo, en las siguientes expresiones las variables $u$, $v$ aparecen vinculada mientras que $l$, $m$ y $n$ cuentan como variables libres}
			\begin{itemize}
				\item{${\displaystyle \sum _{v=l}^{n}\frac{l+v}{n-v}}$} 
				\item{${\displaystyle \prod_{u,v \in S}{\frac{l+v}{n-u}}}$}
				\item{$\bigg({\displaystyle \lim_{v \to n}{\frac{l+1}{n-v}}}\bigg) + m$}
				\item{${\displaystyle \sum _{v=l}^{n}\int_{u=0}^{m-v}{(v-e^{-mu})\,\mathrm{d}u}}$} 
			\end{itemize}
		\end{itemize}
	\item{En el caso de las expresiones asociadas a funciones escritas en la forma $f(t\textsubscript{1}, t\textsubscript{2}, ..., t\textsubscript{n})$ , donde cada variable $t\textsubscript{n}$ entre paréntesis representa un término ... }
		\begin{itemize}
			\item{... una variable aparece libre si y solo si ocurre libre en alguno de los términos $t\textsubscript{1}, t\textsubscript{2}, ..., t\textsubscript{n}$}
			\item{... una variable aparece vinculada si  y solo si ocurre vinculada en alguno de los términos $t\textsubscript{1}, t\textsubscript{2}, ..., t\textsubscript{n}$}
		\end{itemize}
	\item{En el caso de las expresiones asociadas a funciones escritas en la forma $x\textsubscript{1}, x\textsubscript{2}, ..., x\textsubscript{n} \mapsto t$ donde los $x\textsubscript{i}$ son variables y $t$ un término que puede incluir referencias a estas variables ... }
		\begin{itemize}
			\item{... una variable aparece libre si y solo si ocurre libre en $t$ y no es ninguno de los símbolos de las variables que representa los parámetros, en el caso de ejemplo $x\textsubscript{1}, x\textsubscript{2}, ..., x\textsubscript{n}$}
			\item{... una variable aparece vinculada si  y solo si es una de las variables que representa los parámetros, en el caso de ejemplo $x\textsubscript{1}, x\textsubscript{2}, ..., x\textsubscript{n}$, o de lo contrario si ocurre vinculada en el término $t$ }
		\end{itemize}
	\item{Por ejemplo, en las siguientes expresiones las variables $u$, $v$, $w$ aparecen vinculada mientras que $l$, $m$ y $n$ cuentan como variables libres}
		\begin{itemize}
			\item{$u,v \mapsto {\displaystyle \sum _{w=l}^{n} {{n \choose l}u^w v^{n-w}}}$} 
		\end{itemize}
\end{itemize}

\subsection{Reglas de inferencia}

Las demostraaciones matemáticas se basan en una sucesión de resultados intermedios que permiten llegar a una conclusión. Las fórmulas son útiles para describir los resultados obenidos en cada paso. Las reglas de inferencia reflejan las maneras correctas de transformar la estructura de fórmulas que expresan hechos conocidos o demostrados como válidos (antecedente) para arrivar a conclusiones que sean igualmente válidas (consecuente). Es por esto que pueden ser interpretadas como estrategias para demostrar o resolver un problema matemático.

Al abordar las transformaciones que se suceden durante una prueba formal resulta útil utilizar la notación secuente que se escribe de la siguiente manera.

\begin{center}
$A\textsubscript{1}, A\textsubscript{2}, ... A\textsubscript{n} \vdash C$
\end{center}

Esta notación expresa que al asumir la veracidad de los antecedentes (fórmulas) $A\textsubscript{1}, A\textsubscript{2}, ... A\textsubscript{n}$ existe una sucesión de transformaciones de las mismas acordes a las reglas de inferencia, que permiten asegurar la validez de la conclusión expresada en la fórmula $C$ .

Veamos cada regla a continuación para tener una idea más precisa. Comencemos por las más afines a la propia notación secuente para, quizás, entender mejor las implicaciones de su significado.

\begin{ruleinf} \label{relinf_impadd} Teorema de la deducción 
	\begin{equation}
	\begin{gathered}
		\underline {\varphi \vdash \psi }\\
		\varphi \to \psi 
	\end{gathered}
	\end{equation}
\end{ruleinf}


También conocida como \textit{introducción condicional}, es la regla que permite introducir el signo de implicación en la fórmula que queda como consecuente después de la transformación. En términos prácticos esta regla nos da una herramienta para la resolución de problemas, ya que de su planteamiento se constata que si se desea demostrar que \textit{"algo"} implica \textit{"otra cosa"} entonces se puede adoptar como estrategia asumir que ese \textit{"algo"} es verdadero y a partir de eso ir encadenando razonamientos intermedios (basados siempre en las reglas de inferencia) hasta llegar al punto en que la \textit{"otra cosa"} quede demostrada como verdadera. De lograr dicha secuencia de demostración la relación de implicación que buscamos se infiere respaldada por la aplicación directa de esta regla.

\begin{ruleinf} \label{relinf_impadd} Modus ponens
	\begin{equation}
	\begin{gathered} \\
    \varphi \to \psi \\
	\underline {\varphi \quad \quad } \\
	\psi \quad \quad
	\end{gathered}
	\end{equation}
\end{ruleinf}


También conocida como \textit{elimiación condicional} es la regla que permite eliminar la implicación condicional. Interpretándola , nos damos cuent de que no es más que el principio de razonamiento consistete en que si \textit{"algo"} implica \textit{"otra cosa"} entonces desde el mismo momento que se demuestra la validez de ese \textit{"algo"} ya se puede concluir directamente que también es válida la \textit{"otra cosa"}.

\begin{ruleinf} \label{relinf_modustollens} Modus tollens
	\begin{equation}
	\begin{gathered}
	    \varphi \to \psi
		\underline {\lnot \psi \quad \quad \quad }
		\lnot \varphi \quad \quad
	\end{gathered}
	\end{equation}
\end{ruleinf}


Esta regla permite también eliminar la implicación entre dos proposiciones lógicas. Por su naturaleza, fundamenta una estrategia de resolución de problemas, en especial cuando hay que demostrar que algo no se cumple. La misma consiste en asumir como verdadera la proposición objetivo que se quiere probar que es falsa y tratar de demostrar que, de ser ese el caso, entonces se llega a otro resultado de conocida o demostrable falsedad. Al hacerlo, se puede asegurar la invalidez de la proposición objetivo.

\begin{ruleinf} \label{relinf_absurdum} Reductio ad absurdum
	\begin{equation}
	\begin{gathered}
	    \varphi \vdash \psi \\
		\underline {\varphi \vdash \lnot \psi } \\
		\lnot \varphi 
	\end{gathered}
	\end{equation}
	\begin{equation}
	\begin{gathered}
		\lnot\varphi \vdash \psi \\
		\underline {\lnot\varphi \vdash \lnot \psi } \\
		\varphi 
	\end{gathered}
\end{equation}
\end{ruleinf}


La reducción al absurdo es una regla que, al ser interpretada como estrategia de resolución de problemas, es una de las herramientas más poderosas y populares para demostraciones matemáticas. La misma consiste en asumir que \textit{"algo"} se cumple y partir de ese punto con dos líneas de razonamiento, una que demuestre que \textit{"otra cosa"} es verdadera y otra vía que demuestre que esa \textit{"otra cosa"} es falsa. De lograrlo, se puede asegurar que ese \textit{"algo"} que se asumió inicialmete que se cumplía, es por el contrario falso. Hay que mencionar que si al principio se asume que ese \textit{"algo"} que se quiere demostrar es falso, entonces arrivar a la contradicción lo que permite asegurar es que ese \textit{"algo"} sí se cumple.

\begin{ruleinf} \label{relinf_cases} Ánalisis de casos
	\begin{equation}
	\begin{gathered}
		\varphi \rightarrow \chi \\
		\psi \rightarrow \chi  \\
		\underline {\varphi \lor \psi } \\
		\chi 
	\end{gathered}
	\end{equation}
\end{ruleinf}


También conocida como \textit{prueba por casos}, \textit{argumentación basada en casos} o \textit{eliminación de la disjunción} , esta regla es super útil y extremadamete recurrente en las demostraciones matemáticas. Su planteamiento como estrategia de resolución de problemas se basa en un grupo de proposiciones de partida que se puede constatar que en todo momento al menos una es válida. Si se logra demostrar que para cada proposición de partida en ese grupo, si se asume como válida siempre es posible arrivar a la misma conclusión, entonces dicha conclusión es un hecho válido. Muy frecuentemente estas proposiciones de partida consisten en considerar una partición en subconjuntos de los elementos de un conjunto. Por ejemplo, números pares e impares para completar los enteros, o números positivos, negativos y cero para completar los reales. Cada subconjunto sería un caso de análisis.

\begin{ruleinf} \label{relinf_consdilemma} Dilema constructivista
	\begin{equation}
	\begin{gathered}
		\varphi \rightarrow \chi  \\
		\psi \rightarrow \xi  \\
		\underline {\varphi \lor \psi } \\
		\chi \lor \xi 
	\end{gathered}
	\end{equation}
\end{ruleinf}


Esta regla es de uso menos frecuente, pero es interesante y útil en el sentido en que plantea, como estrategia de demostración, que en el problema de probar que bajo cualquier circunstancia al menos una de varias proposiciones siempre es válida, se puede sustituir cualquiera de estas proposiciones por otra más simple o conocida, siempre que la validez de la que se considera como remplazo implique la de la proposición original.

\begin{ruleinf} \label{relinf_cases} Currying
	\begin{equation}
	\begin{gathered}
		\underline {\varphi, \phi  \vdash \psi } \\
		\varphi \vdash \phi \to \psi 
	\end{gathered}
	\end{equation}

	\begin{equation}
		\begin{gathered}
			\underline {\varphi \vdash \phi \to \psi } \\
			\varphi, \phi  \vdash \psi 
		\end{gathered}
	\end{equation}
\end{ruleinf}


Esta regla permite introducir hipótesis intermedias de una demostración como premisas generales. Que quiere decir esto. En una dirección esta regla de inferencia expresa que , si se asume que dos coss y se llega a demostrar una conclusión, entonces es posible asegurar que al asumir una sola de ellas como válida , la conclusión que se puede inferir es que si ocurre la segunda hipótesis será cierta la conclusión. La inferenncia en sentido contrario también es válida. . Es decir , si se parte de una hipótesis que se asume como válida y se demuestra que \textit{"algo"} implica \textit{"otra cosa"} ; esa conclusión permite inferir que al asumir la hipótesis original y ese \textit{"algo"} entonces la \textit{"otra cosa"} que se concluyo anteriormente también es válida.

Como curiosidad el nombbre de la regla se debe a Haskell Curry, notable matemático que, entre sus resultados, logró establecer una relación directa entre programas de computadora. Es por esto que en la programación existe una técnica llamada Currying que permite implementar funciones de varios parámetros en llenguajes de programación funcional como Haskell, donde todas las funciones aceptann un único parámetro. La relación entre ambos escenarios no es fortuita, como tampoco lo es las aplicaciones y semejanzas con temas de la física y otras rmas la matemática superior.

Existen otras reglas de inferencia fundamentales que se mencionan a continuación. Si bien es más complejo interpretarlas en sí mismas como estrategias de solución de problemas, pueden resultar útiles al fundamentar un argumento durante una demostración.

\begin{ruleinf} \label{relinf_add} Adición
	\begin{equation}
	\begin{gathered}
		\underline {\varphi \quad \quad \ \ } \\
		\varphi \lor \psi
	\end{gathered}
	\end{equation}
	\begin{equation}
	\begin{gathered}
		\underline {\psi \quad \quad \ \ } \\
		\varphi \lor \psi
	\end{gathered}
\end{equation}
\end{ruleinf}

\begin{ruleinf} \label{relinf_lorsylo} Silogismo disjuntivo
	\begin{equation}
	\begin{gathered}
		\varphi \lor \psi \\
		\underline {\lnot \varphi \quad \quad } \\
		\psi 
	\end{gathered}
	\end{equation}
	\begin{equation}
	\begin{gathered}
		\varphi \lor \psi \\
		\underline {\lnot \psi \quad \quad } \\
		\varphi
	\end{gathered}
\end{equation}
\end{ruleinf}

\begin{ruleinf} \label{relinf_simpl} Simplificación
	\begin{equation}
	\begin{gathered}
		\underline {\varphi \land \psi } \\
		\varphi 
	\end{gathered}
	\end{equation}
	\begin{equation}
	\begin{gathered}
		\underline {\varphi \land \psi } \\
		\psi 
	\end{gathered}
\end{equation}
\end{ruleinf}

\begin{ruleinf} \label{relinf_impadd} Adjunción
	\begin{equation}
	\begin{gathered}
	    \varphi \quad \quad \ \  \\
		\underline {\psi \quad \quad \ \ } \\
		\varphi \land \psi 
	\end{gathered}
	\end{equation}
\end{ruleinf}

\begin{ruleinf} \label{relinf_equiv} Introducción bicondicional
	\begin{equation}
	\begin{gathered}
		\varphi \to \psi  \\
		\underline {\psi \to \varphi } \\
		\varphi \leftrightarrow \psi 
	\end{gathered}
	\end{equation}
\end{ruleinf}

\begin{ruleinf} \label{relinf_equivdel} Eliminación bicondicional
	\begin{equation}
	\begin{gathered}
		\varphi \leftrightarrow \psi  \\
		\underline {\varphi \quad \quad } \\
		\psi 
	\end{gathered}
	\end{equation}
	\begin{equation}
	\begin{gathered}
		\varphi \leftrightarrow \psi \\
		\underline {\lnot \varphi \quad \quad } \\
		\lnot \psi 
	\end{gathered}
\end{equation}
\end{ruleinf}

\begin{ruleinf} \label{relinf_equivlor} Eliminación bicondicioal por disjunción
	\begin{equation}
	\begin{gathered}
		\varphi \leftrightarrow \psi \\
		\underline {\psi \lor \varphi } \\
		\psi \land \varphi 
	\end{gathered}
	\end{equation}
	\begin{equation}
	\begin{gathered}
		\varphi \leftrightarrow \psi \\
		\underline {\neg\psi \lor \neg\varphi } \\
		\neg\psi \land \neg\varphi 
	\end{gathered}
\end{equation}
\end{ruleinf}

\begin{ruleinf} \label{relinf_equivsym} Simetría bicondicional
	\begin{equation}
	\begin{gathered}
		\underline {\varphi \leftrightarrow \psi } \\
		\psi \leftrightarrow \varphi
	\end{gathered}
	\end{equation}
\end{ruleinf}

\begin{ruleinf} \label{relinf_impadd} Ex contradictione quodlibet
	\begin{equation}
	\begin{gathered}
	    \varphi \\
		\underline {\lnot \varphi } \\
		\psi 
	\end{gathered}
	\end{equation}
\end{ruleinf}


\section{Las relaciones básicas}
\subsection{Igualdad}


\begin{axiom} \label{eq_reflexive} La relación de igualdad es reflexiva
	\begin{equation}
		\forall x\ x=x
	\end{equation}
\end{axiom}

\begin{axiom} \label{eq_subs} Propiedad de sustitución de la relación de igualdad
\begin{center}
	Sea $\phi$ un símbolo que denota una fórmula
\end{center}
	\begin{equation}
		\forall x,y,\phi\ (x=y) \implies (\phi\ = \phi[x \backslash y])
	\end{equation}
\end{axiom}




\begin{theorem} \label{eq_symm} La relación de igualdad es simétrica
	\begin{equation}
		\forall x,y\ (x=y) \implies (y=x)
	\end{equation}
\end{theorem}


\textit{Proof}. 

Es posible constatar que $y = x$ es una  posible sustitución parcial en \cref{eq_reflexive}., por lo tanto , según \cref{eq_subs} se obtiene que

\begin{equation} \label{eq_sym_proof_1}
\begin{gathered}
	\labelcref{eq_subs} , \quad \labelcref{eq_reflexive}\ x=x \quad \vdash \quad (x = y) \implies ((x = x) \implies (y=x))
\end{gathered}
\end{equation}

Basados en \cref{relinf_modustollens}  modus tollens

\begin{equation} \label{eq_sym_proof_2}
\begin{gathered}
	(x = x) \implies (y=x) \\
	\underline{\labelcref{eq_reflexive} \quad x=x \quad \quad \quad} \\
	y = x
\end{gathered}
\end{equation}

\begin{equation}
	\therefore \labelcref{eq_sym_proof_1} , \quad \labelcref{eq_sym_proof_2} \quad \vdash \quad (x = y) \implies (y = x)
\end{equation}

\hfill $\square$

\begin{theorem} \label{eq_trans} La relación de igualdad es transitiva
	\begin{equation}
		\forall x,y,z\ (x=y) \land (y=z) \implies (x=z)
	\end{equation}
\end{theorem}




\section{Los números}
\subsection{Números naturales}


Los elementos del conjunto 

\begin{axiom} \label{nat_zero}
	Elemento nulo
	\begin{equation}
		0 \in \mathbb{N}
	\end{equation}
\end{axiom}
\begin{axiom} \label{nat_closure}
	El conjunto de los números naturales es cerrado con respecto a la relación de sucesor.
	\begin{equation}
		x \in \mathbb{N}\ \implies S(x) \in \mathbb{N}
	\end{equation}
\end{axiom}


Para los números enteros se define una relación de igualdad que, por lo tanto, cumple con \cref{eq_reflexive} , \cref{eq_subs}. En el sistema de Peano \cref{eq_symm} y \cref{eq_trans} son considerados como axiomas. Sin embargo los mismos son demostrales a partir del principio de sustitución \cref{eq_subs}.



\begin{axiom} \label{nat_nosucc} No existe número  natural cuyo sucesor sea el elemento nulo 
	\begin{equation}
		\forall x \in \mathbb{N}\ (0\neq S(x))
	\end{equation}
\end{axiom}
\begin{axiom} \label{nat_inj} La relación de sucesor es inyectiva.
\begin{equation}
	\forall x,y  \in \mathbb{N}\ (S(x)=S(y)\implies x=y)
\end{equation}
\end{axiom}
\begin{axiom} \label{nat_eqclosurej} El conjunto de los números naturales es cerrado con respecto a la relación de igualdad
	\begin{equation}
		\forall x,y \ ((x \in \mathbb{N} \land x=y) \implies y )
	\end{equation}
\end{axiom}


\begin{axiom} \label{nat_sumz} El elemento nulo es la identidad de la operación de suma
	\begin{equation}
		\forall x  \in \mathbb{N}\ (x+0=x)
	\end{equation}
\end{axiom}
\begin{axiom} \label{nat_sumnz} Definición del operador de suma
	\begin{equation}
		\forall x,y  \in \mathbb{N}\ (x+S(y)=S(x+y))
	\end{equation}
\end{axiom}
\begin{axiom} \label{nat_mulz} Muliplicación por elementto nulo
	\begin{equation}
		\forall x  \in \mathbb{N}\ (x\cdot 0=0)
	\end{equation}
\end{axiom}
\begin{axiom} \label{nat_mulnz} Definición del operador de multiplicación
	\begin{equation}
		\forall x,y  \in \mathbb{N}\ (x\cdot S(y)=x\cdot y+x)
	\end{equation}
\end{axiom}


\begin{axiom} \label{nat_ind}
	Principio de inducción matemática
	\begin{center}
		Sea $\phi$ una fórmula con una variable libre $y$ ...
	\end{center}
	\begin{equation}
		{\bigg (}\phi \left[y\backslash0\right] \land \forall n \in \mathbb{N}{\Big (}\phi\left[y\backslash n\right] \implies \phi\left[y \backslash S(n)\right]{\Big )}{\bigg )} \implies \forall x \in \mathbf{N}\ \phi \left[y \backslash x\right]
	\end{equation}
\end{axiom}



\subsection{Números reales}

La definición de de los números reles comienza por las propiedades básicas de las operacione de suma y multiplicación.


\begin{axiom}
	Leyes de asociatividad
	\begin{equation} \label{real_sumssoc}
		\forall x,y,z \in \mathbf{R}\ (x + y) + z = x + (y + z)
	\end{equation}
	\begin{equation} \label{real_mulassoc}
		\forall x,y,z \in \mathbf{R}\ (x \times y) \times z = x \times (y \times z)
	\end{equation}
\end{axiom}

\begin{axiom}
	Leyes de conmutatividad
	\begin{equation} \label{real_sumcomm}
		\forall x,y \in \mathbf{R}\ x + y = y + x
	\end{equation}
	\begin{equation} \label{real_mulcomm}
		\forall x,y \in \mathbf{R}\ (x \times y) = y \times x
	\end{equation}
\end{axiom}

\begin{axiom}
	Ley distributiva
	\begin{equation} \label{real_dist}
		\forall x,y,z \in \mathbf{R}\ x  \times (y + z) = (x \times z) + (y \times z)
	\end{equation}
\end{axiom}

\begin{axiom} \label{real_id}
	Leyes de la identidad
	\begin{equation} \label{eq_real_sumid}
		\forall x \in \mathbf{R}\ x  + 0 = x
	\end{equation}
	\begin{equation} \label{eq_real_mulid}
		\forall x \in \mathbf{R}\ x  \times 1 = x
	\end{equation}
	\begin{equation} \label{real_idneq}
		0 \neq 1
	\end{equation}
\end{axiom}

\begin{axiom} \label{real_inv}
	Leyes del elemento inverso
	\begin{equation} \label{eq_real_suminv}
		\forall x \in \mathbf{R} \exists -x \in \mathbb{R}\ x  + (-x) = 0
	\end{equation}
	\begin{equation} \label{eq_real_mulinv}
		\forall x \in \mathbf{R}\ (x \neq 0 \implies \exists x^{-1} \in \mathbb{R}\ x \times x^{-1} = 1)
	\end{equation}
\end{axiom}






El segundo grupo de axiomas se refiere a la relación de orden total que cumplen los números reales.


\begin{axiom} \label{real_leq_reflexive}
	Reflexividad
	\begin{equation}
		\forall x \in \mathbb{R}\ x \leq x
	\end{equation}
\end{axiom}

\begin{axiom} \label{real_leq_asymm}
	Antisimetría
	\begin{equation}
		\forall x,y \in \mathbb{R}\ (x \leq y) \land (y \leq x) \implies x = y
	\end{equation}
\end{axiom}

\begin{axiom}  \label{real_leq_trans}
	Transitividad
	\begin{equation}
		\forall x,y,z \in \mathbb{R}\ (x \leq y) \land (y \leq z) \implies x \leq z
	\end{equation}
\end{axiom}

\begin{axiom} \label{real_leq_totalorder}
	Orden total
	\begin{equation} 
		\forall x,y \in \mathbb{R}\ (x \leq y) \lor (y \leq x)
	\end{equation}
\end{axiom}


El tercer grupo se refiere a la compatibilidad entre la relación de orden y las operaciones de suma y multtiplicación.

\begin{axiom} \label{real_ops_order}
	Preservación del orden 
	\begin{equation} \label{eq_real_sum_order}
		\forall x,y,z \in \mathbb{R}\ (x \leq y) \implies (x + z \leq y + z)
	\end{equation}

	\begin{equation} \label{eq_real_mul_order}
		\forall x,y \in \mathbb{R}\ (0 \leq x) \land (0 \leq y) \implies (0 \leq x \times y)
	\end{equation}
\end{axiom}


Finalmente, el cuarto grupo de axiomas es el que efectivamente identifica a los números reales entre los demás números. Por ejemplo, los tres grupos de propiedades anteriores también se cumplen para los números racionales, pero no es ese el caso con este cuarto grupo.

\begin{axiom} \label{real_lub}
	Axioma del supremo
	\begin{equation} \label{eq_real_lub}
	\begin{gathered}
		A \subset \mathbb{R}, A \neq \emptyset, \\
		\exists v \in \mathbb{R}\ \forall x \in A\ x \leq v\ \vdash\ \exists u \in A\ (u \leq v) \land (\forall x \in A\ x \leq u)
	\end{gathered}
	\end{equation}
\end{axiom}

\end{document}