\textit{Proof}. 

Es posible constatar que $x = z$ es una  posible sustitución de $y$ por $x$ para la fórmula $y = z$., por lo tanto , según \cref{eq_subs} se obtiene que

\begin{equation} \label{eq_trans_proof_1}
	\begin{gathered}
		\labelcref{eq_subs} \vdash \quad (y = x) \implies ((y = z) \implies (x=z))
	\end{gathered}
\end{equation}

Basados en \cref{eq_symm}  la simetría 

\begin{equation} \label{eq_trans_proof_2}
	\labelcref{eq_trans_proof_1},\ \labelcref{eq_symm} \quad \vdash \quad (x = y) \implies ((y = z) \implies (x=z))
\end{equation}

Aplicando \cref{relinf_currying} uncurrying .

\begin{equation} 
\begin{gathered}
	\underline{\labelcref{eq_trans_proof_2} (x = y) \implies ((y = z) \implies (x=z))} \\
	\underline{\quad \quad \quad \quad\quad x = y, y=z \vdash x=z \quad \quad \quad \quad \quad} \\
	\therefore (x = y) \land (y=z) \implies (x=z)
\end{gathered}
\end{equation}

\hfill $\square$