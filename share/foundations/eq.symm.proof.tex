\textit{Proof}. 

Es posible constatar que $y = x$ es una  posible sustitución parcial en \cref{eq_reflexive}., por lo tanto , según \cref{eq_subs} se obtiene que

\begin{equation} \label{eq_sym_proof_1}
\begin{gathered}
	\labelcref{eq_subs} , \quad \labelcref{eq_reflexive}\ x=x \quad \vdash \quad (x = y) \implies ((x = x) \implies (y=x))
\end{gathered}
\end{equation}

Basados en \cref{relinf_modustollens}  modus tollens

\begin{equation} \label{eq_sym_proof_2}
\begin{gathered}
	(x = x) \implies (y=x) \\
	\underline{\labelcref{eq_reflexive} \quad x=x \quad \quad \quad} \\
	y = x
\end{gathered}
\end{equation}

\begin{equation}
	\therefore \labelcref{eq_sym_proof_1} , \quad \labelcref{eq_sym_proof_2} \quad \vdash \quad (x = y) \implies (y = x)
\end{equation}

\hfill $\square$