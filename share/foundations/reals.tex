\documentclass[12pt]{article}
\usepackage{graphicx}
\usepackage{enumitem}
\usepackage{amsthm}
\usepackage{amsmath}
\usepackage{amssymb}
\usepackage[utf8]{inputenc}
\usepackage{lmodern}
\usepackage[T1]{fontenc}
\usepackage{arcs}
\usepackage{etoolbox}
\usepackage{fancyhdr}
\renewcommand{\qed}{\hfill\square}
\newtheorem{lemma}{Lemma}
\newtheorem{claim}{Claim}
\newtheorem*{problem}{Problem}
\newtheorem{solution}{Solution}
%\usepackage[utf8]{inputenc}
%\usepackage[english]{babel}
\usepackage{tikz}
\usetikzlibrary{matrix}
\pagestyle{fancy}
\lhead{Olimpiada Iberoaméricana de Matemática – Brasil 2023}
\rhead{P2}
\lfoot{P2 @ OIM 2023}
\rfoot{por Olemis Lang (Cuba)}


\begin{document}
\begin{problem}
Sea $\mathbb{Z}$ el conjunto de los enteros. Encuentre todas las funciones $f : \mathbb{Z} \to \mathbb{Z}$ tales que:

\begin{center}
	$2023 f(f(x)) + 2022x^2 = 2022 f(x) + 2023 [f(x)]^2 + 1$
\end{center}

para todo entero $x$.
\end{problem}

\noindent\textbf{Solution 1}\\\\
El enunciado del problema se puede plantear de la siguiente manera:

\begin{equation}
	\forall x \in \mathbb{Z} \mid 2023 f(f(x)) + 2022x^2 = 2022 f(x) + 2023 [f(x)]^2 + 1
\end{equation}

En la solución de este problema se recurrirá a un resultado conocido que se demuestra a continuación 

\begin{lemma}
	Sea $\mathbb{J} \subseteq \mathbb{Z}$ . Sean $m, n \in \mathbb{Z}$ tales que $mcd(m, n) = 1$ y las funciones $f, F : \mathbb{J} \to \mathbb{J}$ entonces
	
	$\forall x \in \mathbb{J} \mid m F(f(x)) = n  F(x) \iff \forall x \in \mathbb{J} \mid F(x) = 0$
\end{lemma}
\textit{Proof}. 
Como punto de partida tenemos la siguiente información

\begin{equation}
	\mathbb{J} \subseteq \mathbb{Z}
\end{equation}
\begin{equation}
	m, n \in \mathbb{Z} \land mcd(m, n) = 1
\end{equation}
\begin{equation}
	f, F : \mathbb{J} \to \mathbb{J}
\end{equation}

Inicialmente hay que constatar que 

\begin{equation}
	\forall x \in \mathbb{J} \mid F(x) = 0 \to (\forall x \in \mathbb{J} \mid m F(f(x)) = 0) \land (\forall x \in \mathbb{J} \mid n F(x) = 0)
\end{equation}
\begin{equation}
	(5) \implies  (\forall x \in \mathbb{J} \mid F(x) = 0) \to (\forall x \in \mathbb{J} \mid m F(f(x)) = n  F(x))
\end{equation}

Supongamos entonces que 

\begin{equation}
	(\forall x \in \mathbb{J} \mid m F(f(x)) = n  F(x)) \land (\exists x\textsubscript{0} \in \mathbb{J} \mid F(x\textsubscript{0}) \neq 0)
\end{equation}

Entonces 

\begin{equation}
	m F(f(x\textsubscript{0})) = n F(x\textsubscript{0})
\end{equation}
\begin{equation}
	(8)  , (3) \implies F(x\textsubscript{0}) \equiv 0 \pmod{m} 
\end{equation}

Tomando (9) como punto de partida es posible suponer que ...

\begin{equation}
\begin{gathered}
	\exists e \in \mathbb{N} ((\forall x \in \mathbb{J} \mid m F(f(x)) = n  F(x) \land \exists x\textsubscript{0} \in \mathbb{J} \mid F(x\textsubscript{0}) \neq 0) \to \\
	(\exists v \in \mathbb{Z} \mid F(x\textsubscript{0}) = m^e  \cdot v) \land (\nexists w \in \mathbb{Z} \mid F(x\textsubscript{0}) = m^{e+1} \cdot w))
\end{gathered}
\end{equation}

Es preciso constatar que 

\begin{equation}
	(8), (7)  \implies F(f(x\textsubscript{0})) \neq 0
\end{equation}

... por tanto las mismas suposicionnes que se hacen en (10) para $x\textsubscript{0}$ se puede hacer para $f(x\textsubscript{0})$ . Sustituyendo se obtiene .

\begin{equation}
	(10) \implies \exists v \in \mathbb{Z} \mid F(x\textsubscript{0}) = m^e \cdot v
\end{equation}
\begin{equation}
	(10), (11) \implies \exists v\textsubscript{1} \in \mathbb{Z} \mid F(f(x\textsubscript{0})) = m^e \cdot v\textsubscript{1}
\end{equation}
\begin{equation}
	(10), (12), (13) \implies m (m^{e} v\textsubscript{1}) = n (m^{e} v)
\end{equation}
\begin{equation}
	(14) \implies m v\textsubscript{1} = n v
\end{equation}
\begin{equation}
	(15), (3) \implies \exists v\textsubscript{2} \in \mathbb{Z} \mid v = m \cdot v\textsubscript{2}
\end{equation}
\begin{equation}
	(10), (16) \implies F(x\textsubscript{0}) = m^{e+1} \cdot v\textsubscript{2}
\end{equation}

La conclusion en (17) contradice la suposiciónn hecha en (10)  de que $\nexists w \in \mathbb{Z} \mid F(x\textsubscript{0}) = m^{e+1} \cdot w$ . Lo que permite concluir que 

\begin{equation}
	(17) \implies (\forall x \in \mathbb{J} \mid m F(f(x)) = n  F(x)) \to \nexists x\textsubscript{0} \in \mathbb{J} \mid F(x\textsubscript{0}) \neq 0
\end{equation}
\begin{equation}
	\therefore (18), (6) \implies (\forall x \in \mathbb{J} \mid m F(f(x)) = n  F(x)) \leftrightarrow (\forall x \in \mathbb{J} \mid F(x) = 0)
\end{equation}

\hfill $\square$

Regresando a considerar el problem original , consideremos reescribir la ecuación funcional de manera siguiente 

\begin{equation}
	(1) \implies \forall x \in \mathbb{Z} \mid 2023 (f(f(x)) - [f(x)]^2 - 1) = 2022 (f(x) - x^2 - 1)
\end{equation}

Teniendo en cuenta que $mcd(2022, 2023) = 1$ y aplicando el teorema antes demostrado considerando las sustituciones $F := x \mapsto f(x) -x^2 - 1$ y $f := f(x)$ se puede concluir que:

\begin{equation}
	(19), (20) \implies \forall x \in \mathbb{Z} \mid f(x) - x^2 - 1 = 0
\end{equation}

... y por tanto la única solución de la ecuación (1) es 

\begin{equation}
	(21) \implies f = x \mapsto x^2 + 1
\end{equation}

\vspace{1cm}
Lo que queda demostrado. \\\\\\

\noindent\textbf{Solution 2}\\\\
El enunciado del problema se puede plantear de la siguiente manera:

\begin{equation}
	\forall x \in \mathbb{Z} \mid 2023 f(f(x)) + 2022x^2 = 2022 f(x) + 2023 [f(x)]^2 + 1
\end{equation}

Notemos que la ecuación dada es equivalente a: 

\begin{equation}
	(23) \implies \forall x \in \mathbb{Z} \mid 2023 (f(f(x)) - (f(x))^2) = 2022 (f(x) - x^2) + 1
\end{equation}

Aanlizando la ecuación módulo $2022$ y módulo $2023$ observamos que 

\begin{equation}
	(24) \implies \forall x \in \mathbb{Z} \mid f(x) - x^2 \equiv 1 \pmod{2023}
\end{equation}
\begin{equation}
	(24) \implies \forall x \in \mathbb{Z} \mid f(f(x)) - f(x)^2 \equiv 1 \pmod{2022}
\end{equation}
\begin{equation}
	(25) \implies \forall x \in \mathbb{Z} \exists k(x) \in \mathbb{Z} \mid f(x) - x^2 = 2023 k(x) + 1
\end{equation}
\begin{equation}
	(26) \implies \forall x \in \mathbb{Z} \exists m(x) \in \mathbb{Z} \mid f(f(x)) - f(x)^2 = 2022 m(x) + 1
\end{equation}

Las dos proposiciones anteriores (27) y (28) sirven de definición para dos funciones auxiliares $k,m : \mathbb{Z} \to \mathbb{Z}$ , que al sustiyuir en la ecuación original se arriva al siguiente resultado:

\begin{equation}
	(23), (27), (28) \implies \forall x \in \mathbb{Z} \mid 2023 (2022 m(x) + 1) = 2022 (2023 k(x) + 1) + 1
\end{equation}
\begin{equation}
	(29) \implies \forall x \in \mathbb{Z} \mid 2023 \cdot 2022 m(x) + 2023 = 2022 \cdot 2023 k(x) + 2023
\end{equation}
\begin{equation}
	(30) \implies \forall x \in \mathbb{Z} \mid m(x) = k(x)
\end{equation}

Concluimos que 

\begin{equation}
\begin{gathered}
	(27), (31), (32) \implies \exists k : \mathbb{Z} \to \mathbb{Z} \forall x \in \mathbb{Z} \mid \\
	(f(x) = x^2 + 2023 k(x) + 1) \\
	(f(f(x)) = f(x)^2 + 2022 k(x) + 1)
\end{gathered}
\end{equation}

Sin embargo , sustituyendo $x := f(x)$ en la ecuación que expresa $f(x)$ en función de $k(x)$ vemos que

\begin{equation}
	(32) \implies \forall x \in \mathbb{Z} \mid f(x)^2 + 1 + 2022 k(x) + 1 = f(x)^2 + 1 + 2023 k(f(x))
\end{equation}
\begin{equation}
	(33) \implies \forall x \in \mathbb{Z} \mid 2022 k(x) = 2023 k(f(x))
\end{equation}

Reutilizando el resultado intermedio de la solución anterior en (19) sustituyendo $F := k$ y $f := f$ .

\begin{equation}
	(34) \implies \forall x \in \mathbb{Z} \mid k(x) = 0
\end{equation}
\begin{equation}
	\therefore  (32), (35) \implies f = x \mapsto x^2 + 1
\end{equation}

\vspace{1cm}
Lo que queda demostrado. \\\\\\



\end{document}