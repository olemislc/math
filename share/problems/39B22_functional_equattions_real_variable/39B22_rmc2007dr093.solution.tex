\noindent\textbf{Solution } \\\\

En lo sucesivo se hará referencia a los siguientes axiomas de los números reales


\begin{axiom} \label{real_leq_reflexive}
	Reflexividad
	\begin{equation}
		\forall x \in \mathbb{R}\ x \leq x
	\end{equation}
\end{axiom}

\begin{axiom} \label{real_leq_asymm}
	Antisimetría
	\begin{equation}
		\forall x,y \in \mathbb{R}\ (x \leq y) \land (y \leq x) \implies x = y
	\end{equation}
\end{axiom}

\begin{axiom}  \label{real_leq_trans}
	Transitividad
	\begin{equation}
		\forall x,y,z \in \mathbb{R}\ (x \leq y) \land (y \leq z) \implies x \leq z
	\end{equation}
\end{axiom}

\begin{axiom} \label{real_leq_totalorder}
	Orden total
	\begin{equation} 
		\forall x,y \in \mathbb{R}\ (x \leq y) \lor (y \leq x)
	\end{equation}
\end{axiom}


El enunciado del problema se traduce a

\begin{equation} \label{eq_39B22_rmc2007dr093_0}
	f: \mathbb{N}^+ \to \mathbb{N}^+
\end{equation}
\begin{equation} \label{eq_39B22_rmc2007dr093_1}
	\forall x,y \in \mathbb{N}^+\ x^2 + f(y) \mid f(x)^2 + y
\end{equation}

Sustituyendo para $x = y = 1$ 

\begin{equation} \label{eq_39B22_rmc2007dr093_2}
	\frac{f(1)^2 + 1}{1+f(1)} \in \mathbb{N}
\end{equation}

... pero hay que constatar que

\begin{equation} \label{eq_39B22_rmc2007dr093_3}
	\frac{f(1)^2 + 1}{1+f(1)} = f(1) - 1 + \frac{2}{1+f(1)}
\end{equation}
\begin{equation} \label{eq_39B22_rmc2007dr093_4}
	\labelcref{eq_39B22_rmc2007dr093_3}, \labelcref{eq_39B22_rmc2007dr093_2} \vdash 1+f(1) \mid 2
\end{equation}
\begin{equation} \label{eq_39B22_rmc2007dr093_5}
	\labelcref{eq_39B22_rmc2007dr093_4}, \labelcref{eq_39B22_rmc2007dr093_0} \vdash f(1) = 1
\end{equation}

Al hallar este valor se puede avanzar en la solución de la siguiente forma

\begin{equation} \label{eq_39B22_rmc2007dr093_6}
	\labelcref{eq_39B22_rmc2007dr093_1} \vdash \forall x,y \in \mathbb{N}^+\ x^2 + f(y) \leq f(x)^2 + y
\end{equation}

Sustituyendo $y=1$

\begin{equation} \label{eq_39B22_rmc2007dr093_7}
	\labelcref{eq_39B22_rmc2007dr093_1}, \labelcref{eq_39B22_rmc2007dr093_5} \vdash \forall x \in \mathbb{N}^+\ x^2 + 1 \leq f(x)^2 + 1
\end{equation}
\begin{equation} \label{eq_39B22_rmc2007dr093_8}
	\labelcref{eq_39B22_rmc2007dr093_7} \vdash \forall x \in \mathbb{N}^+\ f(x) \geq x
\end{equation}

Después de una sustitución similar para $x = 1$ se obtiene que 

\begin{equation} \label{eq_39B22_rmc2007dr093_9}
	\labelcref{eq_39B22_rmc2007dr093_6}, \labelcref{eq_39B22_rmc2007dr093_5} \vdash \forall y \in \mathbb{N}^+\ 1 + f(y) \leq 1 + y
\end{equation}
\begin{equation} \label{eq_39B22_rmc2007dr093_10}
	\labelcref{eq_39B22_rmc2007dr093_9} \vdash \forall y \in \mathbb{N}^+\ f(y) \leq y
\end{equation}

Considerando que la relación de órden es antisimética 

\begin{equation} \label{eq_39B22_rmc2007dr093_11}
	\therefore \labelcref{eq_39B22_rmc2007dr093_10}, \labelcref{eq_39B22_rmc2007dr093_8}, \cref{real_leq_asymm} \vdash \forall x \in \mathbb{N}^+\ f(x) = x
\end{equation}

\vspace{1cm}
Lo que demostra que la única función con esas condiciones es $f = x \mapsto x$. \\\\\\
