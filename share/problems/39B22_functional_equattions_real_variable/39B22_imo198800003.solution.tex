\noindent\textbf{Solution 1} \\\\

El enunciado del problema inndica que

\begin{equation} \label{eq_39B22_imo198800003_1}
	f(1) = 1
\end{equation}
\begin{equation} \label{eq_39B22_imo198800003_2}
f(3) = 3 \\
\end{equation}
\begin{equation} \label{eq_39B22_imo198800003_3}
f(2 \cdot n) = f(n)\\
\end{equation}
\begin{equation} \label{eq_39B22_imo198800003_4}
f(4 \cdot n + 1) = 2 \cdot f(2 \cdot n + 1) - f(n) \\
\end{equation}
\begin{equation} \label{eq_39B22_imo198800003_5}
f(4 \cdot n + 3) = 3 \cdot f(2 \cdot n + 1) - 2 \cdot f(n)
\end{equation}

A partir de estos argumentos iniciales, se corrobora la inducción matemática siguiente

\begin{equation} \label{eq_39B22_imo198800003_6}
	\labelcref{eq_39B22_imo198800003_3} \vdash f(2^k \cdot n) = f(n) \implies  f(2^{k+1} \cdot n) = f(n)
\end{equation}
\begin{equation} \label{eq_39B22_imo198800003_7}
	\labelcref{eq_39B22_imo198800003_6}, \labelcref{eq_39B22_imo198800003_3} \vdash \forall k \in \mathbb{N}\ f(2^k \cdot n) = f(n)
\end{equation}

... y como corolario ...

\begin{equation} \label{eq_39B22_imo198800003_8a}
	\labelcref{eq_39B22_imo198800003_7} \vdash \forall p \in \mathbb{N}\ f(2^p) = f(1)= 1
\end{equation}
\begin{equation} \label{eq_39B22_imo198800003_8}
	\labelcref{eq_39B22_imo198800003_7} \vdash f(4 \cdot n) = f(n)
\end{equation}
\begin{equation} \label{eq_39B22_imo198800003_9}
	\labelcref{eq_39B22_imo198800003_4}, \labelcref{eq_39B22_imo198800003_8} \vdash f(4n + 1)-f(4n) = 2(f(2n + 1) - f(2n))
\end{equation}
\begin{equation} \label{eq_39B22_imo198800003_10}
	\labelcref{eq_39B22_imo198800003_5}, \labelcref{eq_39B22_imo198800003_8} \vdash f(4n + 3)-f(4n+2) = 2(f(2n + 1) - f(2n))
\end{equation}

Además 

\begin{equation}  \label{eq_39B22_imo198800003_11}
		\labelcref{eq_39B22_imo198800003_3} \vdash f(2) = f(1) = 1
\end{equation}

\begin{claim}
	Si $n$ es un número par entonces $f(n+1) - f(n)$ es igual a la mayor potencia de $2$ que es menor que $n$.
\end{claim}

\textit{Proof} Los resultados anteriores permiten plantear una segunda inducción matemática como sigue

\begin{equation}  \label{eq_39B22_imo198800003_12}
	\labelcref{eq_39B22_imo198800003_2}, \labelcref{eq_39B22_imo198800003_11} \vdash \forall n \in \{2,3\} (2 \mid n) \implies f(n+ 1) - f(n) = 2
\end{equation}
\begin{equation}  \label{eq_39B22_imo198800003_13}
\begin{gathered}
	\labelcref{eq_39B22_imo198800003_12}, \labelcref{eq_39B22_imo198800003_9}, \labelcref{eq_39B22_imo198800003_10} \vdash (\forall n \in \{2^p, \cdots, 2^{p+1}- 1\}\ (2 \mid n) \implies f(n+ 1) - f(n) = 2^p) \implies \\
	\forall n \in \{2^{p+1}, \cdots, 2^{p+2}-1\}\ (2 \mid n) \implies f(n+ 1) - f(n) = 2^{p+1} 
\end{gathered}	
\end{equation}
\begin{equation}  \label{eq_39B22_imo198800003_14}
	\labelcref{eq_39B22_imo198800003_12}, \labelcref{eq_39B22_imo198800003_13} \vdash \forall p \in \mathbb{N} \forall n \in \{2^p, \cdots, 2^{p+1}-1\} (2 \mid n) \implies f(n+ 1) - f(n) = 2^p
\end{equation}
\hfill $\square$

\begin{claim}
	Al aplicar la función $f$ a un número entero $n$ se obtiene otro número entero cuya representación binaria es la de $n$, pero en orden inverso.
\end{claim}
\textit{Proof} La afirmación es válida para $1$ \cref{eq_39B22_imo198800003_1}, $2$ \cref{eq_39B22_imo198800003_11} y $3$.\cref{eq_39B22_imo198800003_2}. Para los valores sucesivos solamente es necesario plantear la notación binaria de $n = \sum_{i=0}^{p}{2^i}$ y al suponer la hipótesis válida, constatar que para los números pares del intervalo siguiente se cumple que

\begin{equation}  \label{eq_39B22_imo198800003_15}
	\sum_{i=0}^{p}{a_i 2^{i+1}} + 0 \cdot 2^0 = 2 \cdot \sum_{i=0}^{p}{a_i} 2^i
\end{equation}
\begin{equation}  \label{eq_39B22_imo198800003_16}
	\labelcref{eq_39B22_imo198800003_3}, \labelcref{eq_39B22_imo198800003_15} \vdash f(\sum_{i=0}^{p}{a_i 2^i}) = \sum_{i=0}^{p}{a_{p-i} 2^i} \implies f(\sum_{i=0}^{p}{a_i 2^{i+1}} + 0 \cdot 2^0) =  0 \cdot 2^{p+1} + \sum_{i=0}^{p}{a_{p-i} 2^i} 
\end{equation}

... y para los impares ...

\begin{equation}  \label{eq_39B22_imo198800003_17}
	\sum_{i=0}^{p}{a_i 2^{i+1}} + 1 \cdot 2^0 = 1 + 2 \cdot \sum_{i=0}^{p}{a_i} 2^i
\end{equation}
\begin{equation}  \label{eq_39B22_imo198800003_18}
	\labelcref{eq_39B22_imo198800003_14}, \labelcref{eq_39B22_imo198800003_17}, , \labelcref{eq_39B22_imo198800003_3} \vdash f(\sum_{i=0}^{p}{a_i 2^i}) = \sum_{i=0}^{p}{a_{p-i} 2^i} \implies f(\sum_{i=0}^{p}{a_i 2^{i+1}} + 1 \cdot 2^0) =  1 \cdot 2^{p+1} + \sum_{i=0}^{p}{a_{p-i} 2^i} 
\end{equation}

\hfill $\square$

Por lo tanto $f(n) = n$ si y solo si la representación binaria de $n$ es simetrica.. Solamente resta entonces contar la cantidad de números . Para números de $m$ cifras cualquier permutación de la mitad menos uno de los digitos binarios se replica como espejo a la segunda mitad. Es por esto que en $\interval[open right]{2^p}{2^{p+1}}$ hay $2^{\left[{\frac{m - 1}{2}}\right]}$ números con esta característica. Por lo tanto 

\begin{itemize}
	\item{hay $1$ binario palíndrome en $\interval[open right]{1}{2}$}
	\item{hay $1$ binario palíndrome en $\interval[open right]{2}{4}$}
	\item{hay $2$ binario palíndrome en $\interval[open right]{4}{8}$}
	\item{hay $2$ binario palíndrome en $\interval[open right]{8}{16}$}
	\item{hay $4$ binario palíndrome en $\interval[open right]{16}{32}$}
	\item{hay $4$ binario palíndrome en $\interval[open right]{32}{64}$}
	\item{hay $8$ binario palíndrome en $\interval[open right]{64}{128}$}
	\item{hay $8$ binario palíndrome en $\interval[open right]{128}{256}$}
	\item{hay $16$ binario palíndrome en $\interval[open right]{256}{512}$}
	\item{hay $16$ binario palíndrome en $\interval[open right]{512}{1024}$}
	\item{hay $31$ binario palíndrome en $\interval[open right]{1024}{2024}$}
\end{itemize}

... y el total es $93$

