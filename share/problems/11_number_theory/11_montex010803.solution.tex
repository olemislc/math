\noindent\textbf{Solution } \\\\

Del enunciado del problema 

\begin{equation} \label{eq_11_montex010803_0a}
	m,n \in \mathbb{N}
\end{equation}

Los coeficientes binomiales son números enteros

\begin{equation} \label{eq_11_montex010803_0}
	\forall m,n \in \mathbb{N}\ {n \choose m} \in \mathbb{N}
\end{equation}

Entonces ...

\begin{theorem} \label{tn_div_reflexive}
	La relación de divisibilidad es reflexiva.
	
	\begin{equation} \label{eq_tn_div_reflexive}
		\forall x \in \mathbb{N}\ x \mid x
	\end{equation}
\end{theorem}

\begin{theorem} \label{tn_bezout_mcd} Teorema de Bézout
	Sean $a, b$ números enteros entonces la ecuación $ax + by = n$ tiene solución si y solo si $mcd(a,b)$ divide a $n$.
	\begin{equation} \label{eq_tn_bezout_mcd}
		\forall a,b,n \in \mathbb{N}\ \exists x,y \in \mathbb{N}\ a \cdot x+b \cdot y = n \leftrightarrow mcd(a,b) \mid n
	\end{equation}
\end{theorem}

Combinando los enunciados de los dos teoremas anteriores se obtiene que

\begin{equation} \label{eq_11_montex010803_1}
	\cref{tn_div_reflexive} \vdash mcd(m,n) \mid mcd(m,n)
\end{equation}
\begin{equation} \label{eq_11_montex010803_2}
	\cref{tn_bezout_mcd}, \labelcref{eq_11_montex010803_1} \vdash \exists x,y \in \mathbb{N}\ mcd(m,n) = mx + ny
\end{equation}

... y con este resultado , después de constatar que 

\begin{equation} \label{eq_11_montex010803_3}
	\labelcref{eq_11_montex010803_2} \vdash \frac{mcd(m,n)}{n}{n \choose m} = \left(\frac{mx}{n} + y\right) \frac{n!}{m! (n-m)!}
\end{equation}
\begin{equation} \label{eq_11_montex010803_4}
	\frac{mx}{n} \frac{n!}{m! (n-m)!} = \frac{mx}{n} \frac{n \cdot (n-1)!}{m \cdot (m-1)! (n-m)!} = x \cdot \frac{(n-1)!}{(m-1)!(n-m)!}
\end{equation}
\begin{equation} \label{eq_11_montex010803_5}
	\labelcref{eq_11_montex010803_4}, \labelcref{eq_11_montex010803_5} \vdash \frac{mcd(m,n)}{n} {n \choose m} = x {{m-1} \choose {n-1}} + y {n \choose m}
\end{equation}

Considerando que la expresión original se puede escribir como suma de dos números números naturales y apelando los teoremas básicos que indican que 

\begin{theorem} \label{nat_sum_closure}
	La suma de dos números naturales también es un número natural.
	\begin{equation} \label{eq_nat_sum_closure}
		\forall x,y \in \mathbb{N}\ x+y \in \mathbb{N}
	\end{equation}
\end{theorem}

\begin{theorem} \label{nat_mul_closure}
	El producto de dos números naturales también es un número natural.
	\begin{equation} \label{eq_nat_mul_closure}
		\forall x,y \in \mathbb{N}\ x \cdot y \in \mathbb{N}
	\end{equation}
\end{theorem}

... se puede llegar a la conclusión que

\begin{equation} \label{eq_11_montex010803_5}
	\labelcref{eq_11_montex010803_0a}, \labelcref{eq_11_montex010803_0}, \labelcref{eq_11_montex010803_2}, \labelcref{eq_11_montex010803_5}, \cref{nat_sum_closure}, \cref{nat_mul_closure} \vdash \frac{mcd(m,n)}{n}{n \choose m} \in \mathbb{N}
\end{equation}

\vspace{1cm}
Lo que queda demostrado. \\\\\\
