\begin{problem}
	Un juego de un solo jugador se juega bajo las siguientes reglas. 
	\begin{enumerate}
		\item{Inicialmente hay $10$ columnas verticales de discos con $1$, $2$ ... $10$ discos cada una}
		\item{Se juega por turnos y en cada turno se puede realizar solamente una de las dos jugadas válidas}
		\item{Una jugada válida consiste en seleccionar dos columnas que tengan al menos $2$ discos cada una, se combinan los discos de ambas en una nueva columna y después, encima de ellos, se añaden dos nuevos discos en la misma nueva columa formada}
		\item{La otra jugada válida posible consise en escoger una columna que tenga al menos $4$ discos, de la misma se sacan los dos discos de la parte superior y a partir del resto de los discos de esa columna, se selecciona una cantidad de discos a conveniencia para formar otra nueva columna, manteniendo siempre al menos un disco en la columna donde se empezó a realizar la jugada}
	\end{enumerate}
	
	Un entusiasta jugó una partida hasta llegar al punto en que no pudo realizar más movimientos. ¿Cuáles son las cantidades posibles de columnas que quedan con un solo disco?
\end{problem}
