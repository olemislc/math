
\noindent\textbf{Solution}\\\\

Como el juego es por turnos, describamos el estado antes de hacer una jugada con una tupla del tipo $\langle n, t, c\rangle $ donde

\begin{itemize}
	\item{$n \in \mathbb{N}$ representa la cantidad de jugadas que se han realizado }
	\item{$t \in \mathbb{N}$} sería el número de torres o columnas de discos 
	\item{$c \in \mathbb{N}$ representa la cantidad total de discos, o sea, la suma de la cantidad de discos que hay en todas las columnas}
\end{itemize}

Partiedo de esta notación , e identificando por $\sigma\textsubscript{i}$ la tupla que describe el estado del juego después de $i$ jugadas, la condición inicial del juego puede ser expresada como

\begin{equation} \label{eq_91_obr202300002_1}
	\sigma\textsubscript{0} = \langle 0, 10, 55\rangle 
\end{equation}

Definamos entonces las funciones $j\textsubscript{1}, j\textsubscript{2}: \mathbb{N}^3 \to \mathbb{N}$ que toman como parámetro una configuración del juego y devuelven otra atendiendo a las siguientes definiciones

\begin{equation} \label{eq_91_obr202300002_2}
\begin{gathered}
	j\textsubscript{1}(\langle n, t, c\rangle) = \langle n, t, c\rangle  \mapsto \langle n+1, t-1, c+2\rangle \\
	j\textsubscript{2}(\langle n, t, c\rangle ) = \langle n, t, c\rangle  \mapsto \langle n+1, t+1, c-2\rangle 
\end{gathered}
\end{equation}

Se puede corroborar que $j\textsubscript{1}$ y $j\textsubscript{2}$ se ajustan a las reglas $1$ y $2$ del juego, respectivamente. Por tal razón, la sucesión de jugadas de una partida queda definida entonces de la siguiente manera

\begin{equation} \label{eq_91_obr202300002_3}
	\forall n \in \mathbb{N}\ \sigma\textsubscript{n+1} = 
		\begin{cases}
			j\textsubscript{1}((\sigma\textsubscript{n}) \\
			j\textsubscript{2}((\sigma\textsubscript{n})
		\end{cases}
\end{equation}

Definamos la función $I: \mathbb{N}^3 \to \mathbb{N}$ que evalúa cada configuración de acuerdo a la siguiente definición

\begin{equation} \label{eq_91_obr202300002_4}
	I(\langle n, t, c\rangle ) = 2 t + c
\end{equation}

Entonces , $\forall \sigma\textsubscript{n}$

\begin{equation} \label{eq_91_obr202300002_4a}
	I(\sigma\textsubscript{0}) = 2 \cdot 10 + 55 = 75
\end{equation}
\begin{equation} \label{eq_91_obr202300002_5}
\begin{gathered}
	\labelcref{eq_91_obr202300002_2}, \labelcref{eq_91_obr202300002_4} \vdash I(j\textsubscript{1}(\sigma\textsubscript{n})) = 2 (t - 1) + (c + 2) = 2 t + c \\
	\labelcref{eq_91_obr202300002_2}, \labelcref{eq_91_obr202300002_4} \vdash I(j\textsubscript{2}(\sigma\textsubscript{n})) = 2 (t + 1) + (c - 2) = 2 t + c
\end{gathered}
\end{equation}
\begin{equation} \label{eq_91_obr202300002_6}
	\labelcref{eq_91_obr202300002_5} \vdash \forall n \in \mathbb{N}\  I(\sigma\textsubscript{n+1}) =  I(\sigma\textsubscript{n})
\end{equation}

Es decir , $I$ es un invariante de la posición, por lo tanto, por inducción matemática

\begin{equation} \label{eq_91_obr202300002_7}
	\labelcref{eq_91_obr202300002_4a}, \labelcref{eq_91_obr202300002_6} \vdash \forall n \in \mathbb{N}\ I(\sigma\textsubscript{n}) = 75
\end{equation}

Según los datos, en una configuración final donde no se puedn hacer más jugadas hay $t'$ columnas con exactamente $1$ disco y otra columna adicional con $c'$ discos. Es decir, llegaríamos a una configuración que cumple con

\begin{equation} \label{eq_91_obr202300002_8}
	\exists k \in \mathbb{N}\ \sigma\textsubscript{k} = \langle k, t'+1, t' + c'\rangle 
\end{equation}

Para que no se puedan aplicar ni la jugada $1$ ni la $2$ , se tiene que cumplir que 

\begin{equation} \label{eq_91_obr202300002_9}
	(1 \leq c') \land (c' \leq 3)
\end{equation}

... y entonces ...

\begin{equation} \label{eq_91_obr202300002_10}
	\labelcref{eq_91_obr202300002_7} \vdash I(\sigma\textsubscript{k}) = 75
\end{equation}
\begin{equation} \label{eq_91_obr202300002_11}
	\labelcref{eq_91_obr202300002_8}, \labelcref{eq_91_obr202300002_10} \vdash 2 (t'+1) + t' + c' = 75
\end{equation}
\begin{equation} \label{eq_91_obr202300002_12}
	\labelcref{eq_91_obr202300002_11} \vdash 3 t' = 73 - c'
\end{equation}
\begin{equation} \label{eq_91_obr202300002_12}
	\labelcref{eq_91_obr202300002_12}, \labelcref{eq_91_obr202300002_9}, t' \in \mathbb{N} \vdash (t' = 24) \land (c' = 1)
\end{equation}

Por tanto , como $c' = 1$, la configuración final siempre tiene $t'+ 1 = 25$ columnas con exactamene un disco.

\vspace{1cm}
Lo que queda demostrado. \\\\\\
